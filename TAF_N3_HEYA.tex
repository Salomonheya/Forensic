\documentclass{article}
\usepackage{graphicx} % Required for inserting images

\title{Exercices Chapitre 2}
\author{Salomon Heya}
\date{Jeudi 9 October 2025}

\begin{document}

\section{\underline{Partie 1}}
\subsection{Exercice 1}

\subsection{1-Analyse Comparative des régimes de Vérité}
\paragraph{ -`` L'Ermergence de la trace electronique (1970-1990)'' et ``L'ère du  professionalisme(1990-2000)''}
Le vecteur de dominance de ``L'ermegence de la trace numérique est Rt = 0,7 \newline
Le vecteur de dominance de ``L'ère du professionalisme'' est Rj+p = 0.7

\paragraph{-Comme discontinuité épistémologique nous avons l'absence totale de juridiction ainsi que de formation réelle à la pratique durant la première période qui seront complètement repensées durant la deuxième période  (avènement du web)  avec la cr'éation de l'IOCE}
\paragraph{-Ces ruptures sont nées suite aux faiblesses des sytèmes découvertes les différentes affaires de cybercriminalité, d'où la venue du besoin de création d'un cadre juridique spécifique au cyber-espace et de mise en place de formations professionalles aux métiers et à l'usage du web}

\paragraph{-Il s'agit ici d'une transition progressive}

\subsection{2-Étude de Cas Archéologique Foucaldienne}
\paragraph{-Cas de l'affaire 414s}

Suite à la pénétration dans 60 systèmes dont le Los Alamos National Laboratory, il a été décidé la création d'outils de traçage d'intrusion,\newline ainsi que l'appel à la nécéssité fondamentale qu'est la protection des preuves en particulier numérique

\paragraph{-Il était dicible à cette époque de penser que la seule façon de compromettre un réseau et ses informations n'étaient qu'en l'attaquant physiquement ou qu'il serait obligatoirement nécessaire de se rapprocher du serveur du réseau pour récupérer des informations}

\paragraph{Comme affaire contemporaine nous avons le détournement d'une somme de plusieurs millions par une groupe de hackeurs à la banc Ecobank; le régime de vérité ici est  la technologie avec l'absence de mise en place de protocol capable de répérer, et d'agir rapidement face à ce genre d'incident }

\section{\underline{Partie 3}}
\subsection{Exercice 6}
\paragraph{ Affaire SunDevil}
Nous commençons par la mise en place d'un HoneyPot afin d'attirer un potentiel pirate, suite à cela nous procédons au suivi de son adresse IP, ainsi qu'à une analyse du traffic du réseau pour voir les autres utilisateurs avec qui il a des \newline interactions régulières et qu'on peut ainsi potentiellement catégoriser comme hackeur, suite cela nous cherchons obtenir ses photos soit par les réseaux sociaux soit en nous faisant passer\newline pour une personne qui d'apparence intéressée par la cibe afin qu'après quelques conversations il soit plus enclain à paratger une image de lui afin que nous puissons maintenant collaborer avec les autorités locales pour retrouver son lieu d'habitatiion et ses collèges.




\end{document} 